\documentclass[11pt]{article}

\usepackage[portuguese]{babel}
\usepackage[T1]{fontenc}
\usepackage[utf8]{inputenc}
\usepackage{amsmath}
\usepackage{graphicx}
\usepackage{subfig}
\usepackage[colorinlistoftodos]{todonotes}
\usepackage{listings}
\usepackage{color} 
\usepackage{float}
\usepackage[font={footnotesize}]{caption}
\usepackage{xcolor,colortbl}
\usepackage{array}
\usepackage{fixltx2e}
\setcounter{secnumdepth}{4}

\linespread{1.3}
\usepackage{indentfirst}
\usepackage[top=2cm, bottom=2cm, right=2.5cm, left=2.5cm]{geometry}



\begin{document}


\begin{titlepage}
	\begin{center}
		
		\hfill \break
		\hfill \break
		
		\includegraphics[width=0.3\textwidth]{./logo}~\\[1cm]
		
		\textsc{\Large Mestrado Integrado em Engenharia Electrotécnica e de Computadores}\\[1.5cm]
		\textsc{\huge Sistemas Electrónicos de Processamento de Sinal}\\[0.25cm]
		
		{\huge \bfseries BPSK Modem \\[1.2cm]}
		
		Grupo n.º 2/3 \vspace{0.3cm}
		
		\begin{tabular}{l r}
			André Filipe Barroso Cerqueira \hspace{1mm} & n.º 65144 \\
			Guilherme Branco Teixeira \hspace{1mm} & n.º 70214  \\
			João André Catarino Pereira & n.º 73527
		\end{tabular}
		
		\hfill
		\hfill
		
		segunda-feira 15h30 - 18h30, LE1
		
	
		\vfill
		
		{\large Lisboa, 17 de Abril de 2015} 
		
	\end{center}
\end{titlepage}
\pagenumbering{gobble}
\clearpage

	\footnote{As linhas de código apresentadas durante o relatório têm como objectivo demonstrar a maneira de raciocinar na resolução de problemas, não representando uma cópia exata do código usado em laboratório, podendo até, serem consideradas \textit{pseudo-código}}
	
\tableofcontents
\pagebreak

\clearpage
\pagenumbering{arabic}

\section{Introdução}

\section{Projecto}

\section{Conclusão}

\section{Anexos}
	
\end{document}